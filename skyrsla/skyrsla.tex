\documentclass{article}
\usepackage{graphicx} % Required for inserting images
\usepackage[top=0.9in, bottom=1in, left=1.5in, right=1.5in]{geometry}
\usepackage[utf8]{inputenc}
\usepackage[icelandic]{babel}
\usepackage[T1]{fontenc}
\usepackage[sc]{mathpazo}
\usepackage[parfill]{parskip}
\renewcommand{\baselinestretch}{1.4}
\usepackage{booktabs,tabularx}
\usepackage{multirow}
\usepackage{enumerate}
\usepackage{adjustbox}
\usepackage{multicol}
\usepackage{xcolor}
\usepackage{algpseudocode}
\usepackage{tikz}
\usepackage{nicefrac}
\usepackage{changepage}
\usetikzlibrary{arrows, positioning, calc, graphs}
\usepackage{amsmath, amsfonts, amssymb, amsthm}
\usepackage{graphicx}
\usepackage{tikz}
\usepackage{minted}
\usemintedstyle{manni}
\title{Greining Reiknirita - Lokaverkefni}
\author{Ragnar Björn Ingvarsson, rbi3 \\ Eiður Kristinsson, ???}
\tikzset{->, >=stealth', shorten >=1pt, node distance=2cm,thick, main node/.style={circle,draw,minimum size=3em}}

\begin{document}
\renewcommand\thepage{}

	\maketitle

	\newpage
	\setcounter{page}{1}
	\renewcommand\thepage{\arabic{page}}

	\section{Inngangur}

	Verkefnið snérist aðallega um að vinna úr gögnum frá götukerfi Reykjavíkur 
	og finna t.d. stystu leiðir í kring um borgina, og þar með finna bestu 
	lausn á staðsetningu $k$ hleðslustöðva fyrir bíla til að besta fjarlægð 
	sem þarf að keyra til hleðslustöðvar frá öllum stöðum á höfuðborgarsvæðinu. 

	Fyrir þetta verkefni kláruðum við liði $1-7$, sem fara frá því að lesa inn 
	gögnin, útfæra Dijkstra og $A\star$ reikniritið til að bera saman hvort 
	sé betra, og notfæra okkur þau svo í að finna bestu staðsetningar fyrir 
	2-10 hleðslustöðvar með gráðugu reikniriti.

	\section{Aðferðir}

	Til að byrja með notuðum við \texttt{pandas} safnið til að lesa inn 
	gögnin og vinnum svo úr þeim örlítið til að gera meðhöndlun léttari. Við 
	búum til tvær \texttt{dictionaries}, \texttt{node\_dict} og 
	\texttt{adjacency\_list} sem halda utan um hvern hnút fyrir sig með 
	\texttt{osmid} sem lykil, og lista af leggjum sem tengjast hverjum hnút.

	Til að reikna stystu vegi útfærðum við tvö reiknirit; Dijkstra og $A*$. 
	Útfærsla Dijkstra notast við \texttt{heapdict} sem er töluvert betra 
	heldur en \texttt{heapq} fyrir Dijkstra því hægt er að lækka núverandi 
	lægstu fjarlægð hnútar í stað þess að þurfa að bæta nýrri færslu við, og 
	einnig er það bara almennt hraðara. Reikniritið skilar svo fjarlægðum 
	og fyrri leggjum fyrir alla hnúta. ----vantar um astar----

	Við prófuðum svo föllin og tókum tímann á báðum fyrir $5$ hnúta valda af 
	handahófi og teiknuðum út niðurstöður beggja með \texttt{matplotlib} og 
	\texttt{contextily} til að bera saman.

	Síðan var notað gráðugt reiknirit til að reikna bestu staðsetningar 
	hleðslustöðvanna, sem virkar svo að það lætur alla hnúta merkta með 
	\texttt{primary} koma til greina fyrir hleðslustöðvar og svo, fyrir 
	hvert val á hleðslustöð var reiknað hversu mikið fjarlægðarkostnaður 
	minnkaði fyrir hvern mögulegan hnút. Síðan var hnútnum bætt við í lista 
	sem minnkaði mest heildarkostnað. Þessar niðurstöður voru einnig útfærðar 
	með \texttt{matplotlib} og \texttt{contextily}.

	\section{Niðurstöður}

	ætla að bíða með þennan kafla líklega

	\section{Samantekt og næstu skref}

	þennan líka í bili


\end{document}
